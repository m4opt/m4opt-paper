\documentclass[twocolumn,times]{aastex631}
\usepackage{acronym}

% \received{March 1, 2021}
% \revised{April 1, 2021}
% \accepted{\today}
% \submitjournal{AJ}
\shorttitle{M$^4$OPT}
\shortauthors{Singer et al.}

\acrodef{GW}[GW]{gravitational wave}
\acrodef{KN}[KN]{kilonova}
\acrodefplural{KN}[KNe]{kilonovae}
\acrodef{NUV}[NUV]{near ultraviolet}
\acrodef{FUV}[FUV]{far ultraviolet}
\acrodef{UVEX}[UVEX]{UltraViolet EXplorer}

\begin{document}

\title{Mixed Integer Linear Programming for Time-Domain and Multimessenger Observation Scheduling}

\author[0000-0001-9898-5597]{Leo P. Singer}
\affiliation{Astroparticle Physics Laboratory, NASA Goddard Space Flight Center}
\email{leo.p.singer@nasa.gov}

\author{Friends}

\begin{abstract}
TODO
\end{abstract}

\keywords{Classical Novae (251) --- Ultraviolet astronomy(1736) --- History of astronomy(1868) --- Interdisciplinary astronomy(804)}

\section{Introduction} \label{sec:intro}

\section{Fundamentals of mixed integer programming}

\subsection{Logic to MILP translation dictionary}

\subsection{A simple example}

\subsection{Maximum weighted coverage problem}

\subsection{No overlap constraints}

\section{Progressive elaboration of the problem}

\subsection{Tiling and scheduling without slew constraints}

\section{Case study: GW observations with UVEX}

\Ac{GW} simulations come from \cite{r_weizmann_2024_14142970}.

In Appendix E.2, \citet{2021arXiv211115608K} specifies fiducial parameter ranges for radioactively-powered or shock-powered \ac{KN} models and 90\% credible intervals for the absolute magnitude in each band. These absolute magnitude ranges are reproduced in the Table~\ref{tab:kn-abs-mag} below.

\ac{UVEX} obseves in both the \ac{NUV} and \ac{FUV} filters simultaneously. In orer to achieve a detection in at least one filter, regardless of the model, we should plan obsevations using the fainter of the two models and the brighter of the two bands: the nucleosynthesis-powered model in NUV, with an absolute magnitude range of $[-15.6, -12.4]$. Assuming that this is the 90\% credible interval of a Gaussian distribution, the absolute magnitude has the approximate distribution
%
\begin{equation}
    M_\mathrm{NUV} \sim \mathcal{N}(-14, 1).
\end{equation}

\begin{deluxetable}{lcc}
\tablecaption{\label{tab:kn-abs-mag}Ranges of peak absolute magnitudes of \acp{KN}. Adapted from \citet{2021arXiv211115608K} Appendix E.2.}
\tablehead{
    & \multicolumn2c{absolute magnitude range} \\
    \colhead{Model} & \colhead{NUV} & \colhead{FUV}
}
\startdata
Nucleosynthesis powered & [-15.6, -12.4] & [-17.8, -15.3] \\
Shock powered & [-14.5, -10.2] & [-17.9, -15.0]
\enddata
\end{deluxetable}

\subsection{Comparison with greedy method}

\section{Conclusion}

\begin{acknowledgments}
This work was performed in part at the Aspen Center for Physics, which is supported by National Science Foundation grant PHY-2210452.
\end{acknowledgments}

\vspace{5mm}
\software{
    astropy \citep{2013A&A...558A..33A,2018AJ....156..123A},
    dust\_extinction \citep{2024JOSS....9.7023G},
    dustmaps \citep{2018JOSS....3..695M},
    HEALPix \citep{2005ApJ...622..759G},
    healpy \citep{2019JOSS....4.1298Z},
    ligo.skymap \citep{2016PhRvD..93b4013S,2016ApJ...829L..15S,2016ApJS..226...10S},
    synphot \citep{2018ascl.soft11001S}}

\bibliography{m4opt}{}
\bibliographystyle{aasjournal}

\end{document}
