\documentclass[twocolumn,times]{aastex631}

% \received{March 1, 2021}
% \revised{April 1, 2021}
% \accepted{\today}
% \submitjournal{AJ}
\shorttitle{M$^4$OPT}
\shortauthors{Singer et al.}

\begin{document}

\title{Mixed Integer Linear Programming for Time-Domain and Multimessenger Observation Scheduling}

\author[0000-0001-9898-5597]{Leo P. Singer}
\affiliation{Astroparticle Physics Laboratory, NASA Goddard Space Flight Center}
\email{leo.p.singer@nasa.gov}

\author{Friends}

\begin{abstract}
TODO
\end{abstract}

\keywords{Classical Novae (251) --- Ultraviolet astronomy(1736) --- History of astronomy(1868) --- Interdisciplinary astronomy(804)}

\section{Introduction} \label{sec:intro}

\section{Fundamentals of mixed integer programming}

\subsection{Logic to MILP translation dictionary}

\subsection{A simple example}

\subsection{Maximum weighted coverage problem}

\subsection{No overlap constraints}

\section{Progressive elaboration of the problem}

\subsection{Tiling and scheduling without slew constraints}

\section{Case study: GW observations with UVEX}

\subsection{Comparison with greedy method}

\section{Conclusion}

\begin{acknowledgments}
This work was performed in part at the Aspen Center for Physics, which is supported by National Science Foundation grant PHY-2210452.
\end{acknowledgments}

\vspace{5mm}
\software{astropy \citep{2013A&A...558A..33A,2018AJ....156..123A}}

\bibliography{m4opt}{}
\bibliographystyle{aasjournal}

\end{document}
